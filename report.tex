\documentclass{article}

\usepackage[utf8]{inputenc}
\usepackage{hyperref}
\usepackage{amsmath}
\usepackage{amssymb}

\newcommand{\iassert}[1]{\mathtt{Assert}(#1)}
\newcommand{\icheck}{\mathtt{Check}()}
\newcommand{\icheckpoint}{\mathtt{Checkpoint}()}
\newcommand{\ibacktrack}[1]{\mathtt{Backtrack}(#1)}

\title{Internship Report}
\author{Alban Reynaud}
\date{}

\begin{document}

\maketitle

\begin{abstract}
\end{abstract}

\section{Introduction}
\subsection{Isabelle}
\textit{TODO}

\subsection{SMT Solving and Linear Arithmetic}
\label{smt}
Let $T$ be a quantifier-free theory. A $T$-solver is an algorithm to check if a
finite set of atom of $T$ can be satisfied.

For example, the theory of \textit{linear arithmetic} is composed of atoms (also
called \textit{constraints}) of the form
$$\sum_{i < n} a_i x_i \bowtie b$$
where $a_0, ..., a_{n-1}, b$ are rational constants, $x_0, ..., x_{n-1}$ are
variables and $\bowtie \in \{=,\leqslant,\geqslant,<,>\}$ is a comparison
operator. An assignement $v$ is a function that maps variables to rationals.
If $c = \sum a_i x_i \bowtie b$ is a constraint,  we say that the assignement
$v$ satisfies the constraint c (or $v \vDash c$) if
$$\sum a_i v(x_i) \bowtie b$$.

The other theory we will study is the theory of \textit{mixed-integer linear
arithmetic}. It consists of linear arithmetic problems where some variables are
required to be integer.
%TODO: give an example

Note that we are only interested in the decision problem (\textit{i.e} checking
if a set of constraints admits a satisfying assignment), not the optimization
problem (\textit{i.e} finding an assignment satisfying the constraints that
maximizes a linear objective function). 

Let $\Phi$ be a boolean formula of atoms of the theory $T$. We may want to find
an assignement $v$ that satisfies $\Phi$, or to know if no such assignement
exists. This problem is called \textbf{Satisfiability Modulo Theory} (SMT).

For example, a SMT instance based on linear arithmetic would be:
$$\Phi \equiv (A \vee B \vee C) \wedge (\neg A \vee B) \wedge
              (\neg A \vee C \vee D) \wedge (\neg C \vee D)$$
with
\begin{displaymath}
\begin{array}{lclccc}
  A & \equiv &x  & + & y & \geqslant 3 \\
  B & \equiv &x  &   &   & \leqslant 1 \\
  C & \equiv &   &   & y & \leqslant 1 \\
  D & \equiv &y  & - & x & < 2         \\
\end{array}
\end{displaymath}

Now that SMT and linear arithmetic have been presented, we can move on to the
basics of their resolution.

\subsection{Incremental Interface for SMT solvers}
%TODO: give a chapter number for the book
An efficient procedure to solve SMT problems is DPLL($T$) \cite{Decision2016},
which works as a combination of a SAT-solver and a $T$-solver. Here is a quick
description of such an algorithm:
\begin{itemize}
  \item Replace every atom by a SAT variable to obtain a SAT formula $\Phi$.
    Run a SAT-solver to find a valuation of each variable to 0 or 1 that
    satisfies this SAT-formula.
  \item From this affectation, derive a conjuction of $T$-atoms. Run a
    $T$-solver to find an assignement that satisfies this conjuction.
  \item If an assignment $v$ is found, return $v$
  \item If no assignment is found, from thus conjuction find a contradicting
    subset of atoms. From this subset, derive a new constraint, add it to the
    SAT-formula $\Phi$, and go to the first step.
\end{itemize}

To work efficiently in combination with the SAT-solver, we may assume that the
$T$-solver implements the following interface:
\begin{itemize}
  \item $\iassert{\alpha}$: Asserts the atom $\alpha$. It is added to the set of
    $T$-atoms the should be satisfied.
  \item $\icheck$: Runs a $T$-solver to find an assignment to the set of
    asserted atoms. If such an assignement is found, returns it. Otherwise,
    returns a subset of inconsistent assert atoms.
  \item $\icheckpoint$: Returns a checkpoint $c$ that contains all the necessary
    information to backtrack to the current state.
  \item $\ibacktrack{c}$: Backtrack to the state represented by the checkpoint
    $c$.
\end{itemize}

%TODO: explain why it is called an incremental interface
We will call such an inteface an \textit{incremental interface}.
A similar description can be found in \cite{Dutertre2006} and
\cite{Thiemann2018}.

An example of an execution of DPLL($T$) is given in appendix \ref{dpll}.

Dutertre and de Moura have proposed a Simplex-based solver to solve linear
arithmetic \cite{Dutertre2006}. A partial version of this algorithm has been
formalized in Isabelle by Spacić and Marić \cite{Spacic2012} and extended to
use the incremental interface by Ralph Bottesch, Max Haslbeck and René Thiemann
\cite{Thiemann2018}. The ultimate goal of this internship is to extend the
previous work to solve mixed-integer linear problems.

\bibliographystyle{plain}
\bibliography{sources}

\appendix

\section{Example: an Execution of DPLL($T$)}
\label{dpll}
Suppose that we want to solve the formula $\Phi$ given in section \ref{smt}
using DPLL($T$). First, let us interpret $\Phi$ as a SAT-formula and find an
valuation that satisfies it. For example:
\begin{itemize}
  \item Arbitrarily affects $A$ to 1. Asserts the atom $A$. Get a checkpoints
    $c_1$.
  \item To solve the clause $(\neg A \vee B)$, we must affect $B$ to 1.
    Asserts the atom $B$. Get a checkpoint $c_2$.
  \item Affects $C$ to 1. Asserts the atom $C$. Get a checkpoint $c_3$.
  \item To solve the clause $(\neg C \vee D)$, we must affect $D$ to 1.
    Asserts the atom $D$. Get a checkpoint $c_4$.
\end{itemize}

Now, we have found an valuation that satisfies $\Phi$ interpreted as a
SAT-formula. But we need to check if this valuation is compatible with an
assignment in the theory of linear arithmetic. It means that we need to find an
assignment to the conjuction $A \wedge B \wedge C \wedge D$, which is equivalent
to the system:
\begin{displaymath}
  \left\{
  \begin{array}{ccccr}
    x  & + & y & \geqslant 3 & (A) \\
    x  &   &   & \leqslant 1 & (B) \\
       &   & y & \leqslant 1 & (C) \\
    y  & - & x & < 2         & (D) \\
  \end{array}
  \right.
\end{displaymath}

But this system has no solution. The procedure $\icheck$ may retun that the
constraints $A$, $B$ and $C$ are mutually incompatible. As these three
constraints cannot be satisfied simultaneously, an assignement that satisfies
$\Phi$ must violate at least one of these constraints, so we can deduce that the
clause $(\neg A \vee \neg B \vee \neg C)$ is true. Instead of solving $\Phi$, we
will try to solve $\Phi' = \Phi \wedge (\neg A \vee \neg B \vee \neg C)$. We
need to backtrack, but we can notice that we could bactrack just before
the choice to affect $C$ to 1 was made. So let us backtrack to $c_2$, where only
$A$ and $B$ are affected.

\begin{itemize}
  \item To solve the clause $(\neg A \vee \neg B \vee \neg C)$, we must affect
    $C$ to 0. Asserts the atom $\neg C$. Get a checkpoint $c'_3$.
  \item To solve the clause $(\neg A \vee C \vee D)$, we must affect $D$ to 1.
    Asserts the atom $D$. Get a checkpoint $c'_4$.
\end{itemize}

Again, we have a valuation that satisfies $\Phi'$ interpreted as a
SAT-formula. We have to find an assignement that satisfies the conjuction
$A \wedge B \wedge \neg C \wedge D$, which is equivalent
to the system:
\begin{displaymath}
  \left\{
  \begin{array}{ccccr}
    x  & + & y & \geqslant 3 & (A) \\
    x  &   &   & \leqslant 1 & (B) \\
       &   & y & > 1         & (\neg C) \\
    y  & - & x & < 2         & (D) \\
  \end{array}
  \right.
\end{displaymath}

$\icheck$ may return the assignement $(x=1, y=2)$. Finally, the formula $\Phi$
is satisfiable, and $(x=1, y=h) \vDash \Phi$.

\section{Resolution of Mixed-Integer Linear Problems}

\end{document}
